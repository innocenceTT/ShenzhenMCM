% !Mode:: "TeX:UTF-8"
%!TEX program  = xelatex

\documentclass[bwprint]{cumcmthesis} 


\title{代谢综合征的检查、影响因素以及预测模型}

\tihao{A}
\baominghao{54321}
\schoolname{XX大学}
\membera{小米}
\memberb{向左}
\memberc{哈哈}
\supervisor{老师}
\yearinput{2015}
\monthinput{08}
\dayinput{22}
\begin{document}
 \maketitle
 \begin{abstract}
 目前住宅空间的紧张导致越来越多的折叠家具的出现。某公司设计制作了一款折叠桌以满足市场需要。
 以此折叠桌为背景提出了三个问题,本文运用几何知识、非线性约束优化模型等方法成功解决了这三个问题,得到了折叠桌动态过程的描述方程以及在给定条件下怎样选择最优设计加工参数,并针对任意形状的桌面边缘线等给出了我们的设计。

针对问题一,根据木板尺寸、木条宽度,首先确定木条根数为19根,接着,根据桌子是前后左右对称的结构,我们只以桌子的四分之一为研究对象,运用空间几何的相关知识关系,推导并建立了几何模型。接着用MATLAB软件编程,绘制出折叠桌动态变化过程图。然后求出折叠桌各木条相对桌面的角度、各木条长度、各木条的开槽长度等数据,相关结果见表1。然后建立相应的三维坐标系,求出桌角各端点坐标,绘出桌角边缘线曲线图,并用MATLAB工具箱作拟合,求出桌角边缘线的函数关系式,并对拟合效果做分析(见表3)。

针对问题二,在折叠桌高度、桌面直径已知情况下,综合考虑桌子稳固性、加工方便、用材最少三个方面因素,我们运用材料力学等相关知识,对折叠桌作受力分析,确定稳固性、加工方便、用材最少三个方面因素间的相互制约关系,建立非线性优化模型。用lingo软件编程,求出对于高70 cm,桌面直径80 cm的折叠桌,平板尺寸172.24cm×80cm×3cm、钢筋位置在桌腿上距离铰链46.13cm处、各木条的开槽长度(见表3)、最长木条(桌脚)与水平面夹角71.934°。

针对问题三,对任意给出的桌面边缘线(f(x)),不妨假定曲线是对称的(否则,桌子的稳定性难以保证),将对称轴上n等份,依照等份点沿着木板较长方向平行的方向下料,则这些点即是铰接处到木板中垂线(相对于木板长方向)的距离。然后修改问题二建立的优化模型,用lingo软件编程,得到最优设计加工参数(平板尺寸、钢筋位置、开槽长度等)。最后,我们根据所建立的模型,设计了一个桌面边缘线为椭圆的折叠桌,并且给出了8个动态变化过程图(见图10)和其具体设计加工参数(见表5)。

最后,对所建立的模型和求解方法的优缺点给出了客观的评价,并指出了改进的方法。

\keywords{折叠桌\quad  曲线拟合\quad   非线性优化模型\quad  受力分析}
\end{abstract}

\section{问题重述}
\subsection{引言}
当前的生命科学技术已经使得我们可以解析人类的遗传密码——基因序列。人与人之间基因序列的差异,即基因变异,影响着每个人罹患代谢综合疾病的风险高低,也影响着不同非遗传因素在每个人身上的具体作用效果。这也是为何代谢综合征在具有血缘关系的亲属之间有较高的发病关联的原因。此外,对人体生理运行动态特征,如基因的表达、各类小分子含量乃至于与人类紧密关联的微生物菌群的变化等,目前亦可以定量测量,从而实现了对人体当前运行状态动态监控。另一方面,各类自然和社会环境因素、生活方式等因素会对这些指征可能产生短期到长期的影响,最终形成多种外部表型特征,目前也可以通过各类移动医疗和健康设备加以记录,成为医疗及体检机构的临床检测数据。

代谢综合征(metabolic syndrome)是指多种代谢成分异常聚集的病理状态,是一组复杂的代谢紊乱症候群。引起代谢综合征的危险因子主要包括高血压、血脂异常、糖尿病、肥胖以及高尿酸与凝血因子不正常等。它是导致心脑血管疾病的危险因素,其集簇发生可能与胰岛素抵抗有关。随着近几十年来物质资源的逐渐丰富,我国代谢综合征及后续心血管疾病及糖尿病的发展迅速蔓延,成为人民群众健康的主要威胁之一。目前,尽管代谢综合征及其后续关联疾病的发病机制尚未阐明,但已经表明和遗传、环境、心理、生活方式、及年龄有关。

人体作为一个非线性复杂系统,其慢性疾病,特别是代谢综合征的发展,也是一个长期过程;而最终在临床上被诊断为代谢综合征,已经是这个慢性发展过程的结果。而我们平时所能够采取的健康和疾病预防手段,通常也只是针对于普适人群的平均推荐,不一定适合于每一个人。
\subsection{问题的提出}
本文将基于一个人类群体(100人)的临床检测数据、基因组数据、表观基因组数据、转录组数据、蛋白质组数据以及代谢组数据,在早期对人体各类从内部到外部的因素进行测量、分析,构建一个早期的趋势预测模型,完成以下任务,以明晰这个复杂系统的具体问题所在,完成对每个人的个性化预防干预:

a)参考NCBI,EBI,DDBJ等公开数据库中的生物分子相互作用和基因通路信息,构建人类生命量化的动态模型;

b)结合临床检测数据,哪些因素(数据特征或相互作用网络)是代谢综合症(一种临床诊断结论)的关键通路;

c)若给定一个新的人类群体数据集(10人),包含了每个个体的基因组、表观基因组、转录组、蛋白质组和(或)代谢组的部分测量,则这些人有多大的代谢综合症风险,造成他们的代谢综合症风险的主要因素分别是什么?

\section{模型的假设}
\begin{itemize}
\item 忽略实际加工误差对设计的影响;
\item 木条与圆桌面之间的交接处缝隙较小,可忽略;
\item 钢筋强度足够大,不弯曲;
\item 假设地面平整。
\end{itemize}
\section{符号说明}
\begin{tabular}{cc}
 \hline
 \makebox[0.4\textwidth][c]{符号}	&  \makebox[0.5\textwidth][c]{意义} \\ \hline
 D	    & 木条宽度(cm) \\ \hline
 L	    & 木板长度(cm)  \\ \hline
 W	    & 木板宽度(cm)  \\ \hline
 N	    & 第n根木条  \\ \hline
 T	    & 木条根数  \\ \hline
 H	    & 桌子高度(cm)  \\ \hline
 R	    & 桌子半径(cm)  \\ \hline
 R	    & 桌子直径(cm)  \\ \hline
\end{tabular}
\section{问题分析}
代谢综合征,是一组以向心性肥胖、甘油三脂水平升高、空腹血糖水平升高、血压升高、及高密度脂蛋白降低等多种代谢成分异常聚集为表征的病理状态,如今已经呈现全球流行趋势,成为世界范围内主要的公共卫生眺赵之一。进行疾病的病因分析常选用logistic回归分析,它是研究分类反应变量与多个影响因素之间关系的一种多变量影响方法。
\subsection{问题一分析}
题目要求建立模型描述折叠桌的动态变化图,由于在折叠时用力大小的不同,我们不能描述在某一时刻折叠桌的具体形态,但我们可以用每根木条的角度变化来描述折叠桌的动态变化。首先,我们知道折叠桌前后左右对称,我们可以运用几何知识求出四分之一木条的角度变化。最后,根据初始时刻和最终形态两种状态求出桌腿木条开槽的长度。
<<<<<<< HEAD
\subsection{问题二分析}
根据附录的相关背景,引起代谢综合征的危险因子主要包括高血压、血脂异常、糖尿病、肥胖以及高尿酸与凝血因子不正常等。
=======
\subsection{问题b分析}
该问题要求结合临床检测数据,分析可能的致病因素与代谢综合征的相关性,找出代谢综合征的关键通路。

在生物——社会——心理医学模式下寻找因素,尤其是慢性病的危险因素,应从多方面进行探索和分析。从logistic回归系数和OR的关系可见,logistic回归模型是分析分类反应变量与多个解释变量相互关系的强有力的工具,可对代谢综合征的影响因素进行多因素分析,从诸多可能因素中筛选出危险因素作为关键通路的备选。

此外,在病理研究中,研究者在对其致病因素的作用进行分析时,会受到混杂因素的干扰。如果对混杂因素不加以控制,就会使结果产生偏倚。控制混杂因素主要从两个方面,一是研究设计时进行控制,即通过匹配设计或分层抽样使分组的混杂因素均衡;二是统计分析时,如果混杂因素较少,才可采用Mantel-Haenszel进行分层分析。而本文的侧重点在于后者的影响。最好的方法就是采用logistic回归的方法对数据进行分析。logistic回归不仅能充分利用临床检测数据,还可以有效地实现对混杂因素影响的控制,得到校正后0R的估计值和可信区间。从而校正混杂因素影响后的代谢综合征与危险因素的关系作出分析,得到更准确的结论。
>>>>>>> refs/remotes/MrGodfrey/master
\subsection{问题三分析}
题目要求制作软件的意思就是客户给定折叠桌高度、桌面边缘线的形状大小和桌脚边缘线的大致形状,将这些信息输入程序就得到客户想要的桌子。我们在求解最优设计加工参数时,自行给定桌面边缘线形状(椭圆、相交圆等),桌脚边缘线形状,折叠桌高度,应用第二问的非线性规划模型,用MATLAB软件绘制折叠桌截面图,得到自己设计的创意平板折叠桌。

\section{模型的假设与符号说明}
\subsection{模型的假设与使用}
\subsection{符号的说明与使用}
\section{模型的准备}
由于现阶段医学对代谢综合征的研究提供的数据不足,故本文采用对糖尿病研究得到的数据代替。日后进行更深一步的医学研究后,可再使用对代谢综合征的研究数据代入,即可得到准确的结果。

数据前十行如下:

参数的意义如下:

\section{模型的建立与求解}
\subsection{问题b的模型建立与求解}
\subsubsection{模型介绍}
在医学研究中常用logistic模型研究疾病与其影响因素的关系。
\[ Y=\begin{cases}
1 &\text{出现阳性结果}\\
0 &\text{出现阴性结果}
\end{cases} \]
设对翻译变量Y有影响的因素有m个,称为自变量,记为$ X_{1},X_{2},\cdots,X_{m} $。
在m个自变量的作用下出现阳性结果的条件概率记为$ P=P(Y=1|X_{1},X_{2}\cdots,X_{m}) $,考虑下式
\begin{equation}
\label{logistic}
P=\dfrac{\exp(\beta_{0}+\beta_{1}X_{1}+\beta_{2}X_{2}+\ldots+\beta_{m}X_{m})}{1+\exp(\beta_{0}+\beta_{1}X_{1}+\beta_{2}X_{2}+\ldots+\beta_{m}X_{m})}
\end{equation}

称式\eqref{logistic}所定义的模型为logistic回归模型。

对模型中的自变量$ X_{i} $,若$ X_{i} $为非数值型变量,则用特殊的方法将其数值化。
\subsubsection{回归系数的估计与检验}
\paragraph{回归系数的估计}
logistic回归模型的参数的估计通常采用的是最大似然估计法,其统计原理为:对n例观察样本建立似然函数\[ L=\prod_{i=1}^{n}P_{i}^{Y_{i}}\quad i=1,2,\cdots,n \]
式中$ P_{i} =P(Y_{i}=1|X_{1},X_{2},\ldots,X_{n})$表示第i例观察对象在自变量的作用下阳性结果发生的概率,如果实际出现的是阳性结果,取$ Y_{i}=1 $。最大似然函数$ L(\beta) $达到最大值。

这一计算过程需要借助统计软件,我们采用SPSS~22.0

我们对预先给的数据的计算结果如下:

\paragraph{回归参数的检验}
获得logistic回归模型的参数估计后,需要对你和的logistic回归模型进行检验。
\subparagraph{对模型回归系数整体的检验}
\subparagraph{对模型中回归系数进行检验}
各因素的回归系数估计及检验结果如下:

表中第2列为回归系数的最大似然估计值,第3列为估计值的标准误差,第4列为Wald检验的统计量值,第5列为Wald$ \chi^{2} $检验的P值,第6列为标准化回归系数的最大似然估计值,第7列为OR估计值,第8、9列为优势比OR的95\% 可信区间的上、下限。

\paragraph{模型参数的流行病学意义}
参数$ \beta_{0} $的意义:当各相关因素不存在作用时,发病与不发病的概率之比的自然对数,它反映了疾病的本底状态。

参数$ \beta_{j} $的意义:设自变量$ X_{j} $的两个取值为$ x_{j}=e_{j} $和$ X_{j}=e_{0} $,假定其他变量的取值保持不变,由流行病学的知识可知
\[ \ln OR_{j}=\ln[\dfrac{P_{1}/(1-P_{1})}{P_{0}/(1-p_{0})}]=\beta_{j}(e_{1}-e_{0}) \]\\
取对数后可得
\begin{equation}
OR_{j}=\exp [\beta_{j}(e_{1}-e_{0})]
\end{equation}

$ OR_{j} $称为调整优势比,表示扣除了其他自变量影响后,自变量$ X_{j} $的作用。如果$ x_{j} $仅取二值
\[ X_{j}=\begin{cases}
1 & \text{暴露}\\
0 & \text{非暴露}
\end{cases} \]
则暴露组与非暴露组出现阳性结果的优势比为
\begin{equation}
OR_{j}=\exp\beta_{j} 
\end{equation}
由指数函数的性质可得。当$ \beta_{j}=o $时,$ OR_{j}=1 $,说明自变量$ X_{j} $对是否出现阳性结果不存在影响;当$ \beta\neq0 $时,$ OR_{j}\neq1 $,说明$ X_{j} $可能是危险因子或是保护因子。

优势比$ OR $的区间估计计算公式为
\begin{equation}
b_{j}\pm U_{\alpha /2}SE(b_{j}) 
\end{equation}
其中$ SE(B_{j}) $为回归系数$ b_{j} $渐进标准误。$ U_{\alpha /2} $为标准正态分布的界值。$ OR $的$ 100(1-\alpha)\% $可信区间为
\begin{equation}
\exp(b_{j}\pm U_{\alpha /2}SE(b_{j})) 
\end{equation}

回归系数绝对值的大小可反映自变量对模型的贡献大小,当各自变量的单位不同时,可比较标准化的回归系数$ \beta_{j} $的估计值$ b_{j}^{'} $的绝对值,$ b_{j}^{'} $计算公式为:
\begin{equation}
b_{j}^{'}=b_{j}S_{j}/(\pi/\sqrt{3})=0.5513b_{j}S_{j}
\end{equation}
其中$ S_{j} $为自变量$ X_{j} $标准差。


问题三流程图:
\begin{figure}[!h]
\centering
%\includegraphics[width=\textwidth]{1.png}
\caption{问题三流程图}
\end{figure}
\begin{thebibliography}{9}
 \bibitem{bib:one} ....
 \bibitem{bib:two} ....
\end{thebibliography}
\appendix
\section{我的 MATLAB 源程序}


\end{document} 