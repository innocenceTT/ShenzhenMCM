% !Mode:: "TeX:UTF-8"
%!TEX program  = xelatex

\documentclass[bwprint]{cumcmthesis} 

\title{代谢综合征的检测、影响因素以及预测模型}
\tihao{A}
\baominghao{4321}
\schoolname{XX大学}
\membera{小米}
\memberb{向左}
\memberc{哈哈}
\supervisor{老师}
\yearinput{2015}
\monthinput{08}
\dayinput{22}
\begin{document}
 \maketitle
 \begin{abstract}
 目前住宅空间的紧张导致越来越多的折叠家具的出现。某公司设计制作了一款折叠桌以满足市场需要。
 以此折叠桌为背景提出了三个问题,本文运用几何知识、非线性约束优化模型等方法成功解决了这三个问题,得到了折叠桌动态过程的描述方程以及在给定条件下怎样选择最优设计加工参数,并针对任意形状的桌面边缘线等给出了我们的设计。

针对问题一,根据木板尺寸、木条宽度,首先确定木条根数为19根,接着,根据桌子是前后左右对称的结构,我们只以桌子的四分之一为研究对象,运用空间几何的相关知识关系,推导并建立了几何模型。接着用MATLAB软件编程,绘制出折叠桌动态变化过程图。然后求出折叠桌各木条相对桌面的角度、各木条长度、各木条的开槽长度等数据,相关结果见表1。然后建立相应的三维坐标系,求出桌角各端点坐标,绘出桌角边缘线曲线图,并用MATLAB工具箱作拟合,求出桌角边缘线的函数关系式,并对拟合效果做分析(见表3)。

针对问题二,在折叠桌高度、桌面直径已知情况下,综合考虑桌子稳固性、加工方便、用材最少三个方面因素,我们运用材料力学等相关知识,对折叠桌作受力分析,确定稳固性、加工方便、用材最少三个方面因素间的相互制约关系,建立非线性优化模型。用lingo软件编程,求出对于高70 cm,桌面直径80 cm的折叠桌,平板尺寸172.24cm×80cm×3cm、钢筋位置在桌腿上距离铰链46.13cm处、各木条的开槽长度(见表3)、最长木条(桌脚)与水平面夹角71.934°。

针对问题三,对任意给出的桌面边缘线(f(x)),不妨假定曲线是对称的(否则,桌子的稳定性难以保证),将对称轴上n等份,依照等份点沿着木板较长方向平行的方向下料,则这些点即是铰接处到木板中垂线(相对于木板长方向)的距离。然后修改问题二建立的优化模型,用lingo软件编程,得到最优设计加工参数(平板尺寸、钢筋位置、开槽长度等)。最后,我们根据所建立的模型,设计了一个桌面边缘线为椭圆的折叠桌,并且给出了8个动态变化过程图(见图10)和其具体设计加工参数(见表5)。

最后,对所建立的模型和求解方法的优缺点给出了客观的评价,并指出了改进的方法。

\keywords{折叠桌\quad  曲线拟合\quad   非线性优化模型\quad  受力分析}
\end{abstract}

\section{问题重述}
\subsection{引言}
创意平板折叠桌注重于表达木制品的优雅和设计师所想要强调的自动化与功能性。为了增大有效使用面积。设计师以长方形木板的宽为直径截取了一个圆形作为桌面,又将木板剩余的面积切割成了若干个长短不一的木条,每根木条的长度为平板宽到圆上一点的距离,分别用两根钢筋贯穿两侧的木条,使用者只需提起木板的两侧,便可以在重力的作用下达到自动升起的效果,相互对称的木条宛如下垂的桌布,精密的制作工艺配以质朴的木材,让这件工艺品看起来就像是工业革命时期的机器。
\subsection{问题的提出}
围绕创意平板折叠桌的动态变化过程、设计加工参数,本文依次提出如下问题:

(1)给定长方形平板尺寸 ($120 cm \times 50 cm \times 3 cm$),每根木条宽度(2.5 cm),连接桌腿木条的钢筋的位置,折叠后桌子的高度(53 cm)。要求建立模型描述此折叠桌的动态变化过程,并在此基础上给出此折叠桌的设计加工参数和桌脚边缘线的数学描述。

(2)折叠桌的设计应做到产品稳固性好、加工方便、用材最少。对于任意给定的折叠桌高度和圆形桌面直径的设计要求,讨论长方形平板材料和折叠桌的最优设计加工参数,例如,平板尺寸、钢筋位置、开槽长度等。对于桌高70 cm,桌面直径80 cm的情形,确定最优设计加工参数。

(3)给出软件设计的数学模型,可以根据客户任意设定的折叠桌高度、桌面边缘线的形状大小和桌脚边缘线的大致形状,给出所需平板材料的形状尺寸和切实可行的最优设计加工参数,使得生产的折叠桌尽可能接近客户所期望的形状,并根据所建立的模型给出几个设计的创意平板折叠桌。要求给出相应的设计加工参数,画出至少8张动态变化过程的示意图。

\section{模型的假设}
\begin{itemize}
\item 忽略实际加工误差对设计的影响;
\item 木条与圆桌面之间的交接处缝隙较小,可忽略;
\item 钢筋强度足够大,不弯曲;
\item 假设地面平整。
\end{itemize}
\section{符号说明}
\begin{tabular}{cc}
 \hline
 \makebox[0.4\textwidth][c]{符号}	&  \makebox[0.5\textwidth][c]{意义} \\ \hline
 D	    & 木条宽度(cm) \\ \hline
 L	    & 木板长度(cm)  \\ \hline
 W	    & 木板宽度(cm)  \\ \hline
 N	    & 第n根木条  \\ \hline
 T	    & 木条根数  \\ \hline
 H	    & 桌子高度(cm)  \\ \hline
 R	    & 桌子半径(cm)  \\ \hline
 R	    & 桌子直径(cm)  \\ \hline
\end{tabular}
\section{问题分析}
代谢综合征,是一组以向心性肥胖、甘油三脂水平升高、空腹血糖水平升高、血压升高、及高密度脂蛋白降低等多种代谢成分异常聚集为表征的病理状态,如今已经呈现全球流行趋势,成为世界范围内主要的公共卫生眺赵之一。
\subsection{问题一分析}
题目要求建立模型描述折叠桌的动态变化图,由于在折叠时用力大小的不同,我们不能描述在某一时刻折叠桌的具体形态,但我们可以用每根木条的角度变化来描述折叠桌的动态变化。首先,我们知道折叠桌前后左右对称,我们可以运用几何知识求出四分之一木条的角度变化。最后,根据初始时刻和最终形态两种状态求出桌腿木条开槽的长度。
\subsection{问题二分析}

\subsection{问题三分析}
题目要求制作软件的意思就是客户给定折叠桌高度、桌面边缘线的形状大小和桌脚边缘线的大致形状,将这些信息输入程序就得到客户想要的桌子。我们在求解最优设计加工参数时,自行给定桌面边缘线形状(椭圆、相交圆等),桌脚边缘线形状,折叠桌高度,应用第二问的非线性规划模型,用MATLAB软件绘制折叠桌截面图,得到自己设计的创意平板折叠桌。

问题三流程图:
\begin{figure}[!h]
\centering
%\includegraphics[width=\textwidth]{1.png}
\caption{问题三流程图}
\end{figure}
\begin{thebibliography}{9}
 \bibitem{bib:one} ....
 \bibitem{bib:two} ....
\end{thebibliography}
\appendix
\section{我的 MATLAB 源程序}


\end{document} 